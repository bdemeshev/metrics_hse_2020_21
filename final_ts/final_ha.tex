\documentclass[12pt]{article}

\usepackage{tikz} % картинки в tikz
\usepackage{microtype} % свешивание пунктуации

\usepackage{array} % для столбцов фиксированной ширины

\usepackage{indentfirst} % отступ в первом параграфе

\usepackage{sectsty} % для центрирования названий частей
\allsectionsfont{\centering}

\usepackage{amsmath} % куча стандартных математических плюшек

\usepackage{comment}
\usepackage{amsfonts}

\usepackage[top=1.5cm, left=1cm, right=1cm, bottom=1.5cm]{geometry} % размер текста на странице

\usepackage{lastpage} % чтобы узнать номер последней страницы

\usepackage{enumitem} % дополнительные плюшки для списков
%  например \begin{enumerate}[resume] позволяет продолжить нумерацию в новом списке
\usepackage{caption}

\usepackage{hyperref} % гиперссылки

\usepackage{multicol} % текст в несколько столбцов


\usepackage{fancyhdr} % весёлые колонтитулы
\pagestyle{fancy}
\lhead{Домашка}
\rhead{Эконометрика}
\lfoot{}
\cfoot{}
\rfoot{}
\renewcommand{\headrulewidth}{0.4pt}
\renewcommand{\footrulewidth}{0.4pt}



\usepackage{todonotes} % для вставки в документ заметок о том, что осталось сделать
% \todo{Здесь надо коэффициенты исправить}
% \missingfigure{Здесь будет Последний день Помпеи}
% \listoftodos --- печатает все поставленные \todo'шки


% более красивые таблицы
\usepackage{booktabs}
% заповеди из докупентации:
% 1. Не используйте вертикальные линни
% 2. Не используйте двойные линии
% 3. Единицы измерения - в шапку таблицы
% 4. Не сокращайте .1 вместо 0.1
% 5. Повторяющееся значение повторяйте, а не говорите "то же"


\usepackage{fontspec}
\usepackage{polyglossia}

\setmainlanguage{russian}
\setotherlanguages{english}

% download "Linux Libertine" fonts:
% http://www.linuxlibertine.org/index.php?id=91&L=1
\setmainfont{Linux Libertine O} % or Helvetica, Arial, Cambria
% why do we need \newfontfamily:
% http://tex.stackexchange.com/questions/91507/
\newfontfamily{\cyrillicfonttt}{Linux Libertine O}

\AddEnumerateCounter{\asbuk}{\russian@alph}{щ} % для списков с русскими буквами
\setlist[enumerate, 2]{label=\asbuk*),ref=\asbuk*}
%\setlist[enumerate, 1]{label=\asbuk*),ref=\asbuk*}


%% эконометрические сокращения
\DeclareMathOperator{\Cov}{Cov}
\DeclareMathOperator*{\plim}{plim}
\DeclareMathOperator{\Corr}{Corr}
\DeclareMathOperator{\Var}{Var}
\DeclareMathOperator{\E}{E}
\let\P\relax
\DeclareMathOperator{\P}{P}
\DeclareMathOperator \hVar{\widehat{\Var}}
\DeclareMathOperator \hCorr{\widehat{\Corr}}
\DeclareMathOperator \hCov{\widehat{\Cov}}

\newcommand \hb{\hat{\beta}}
\newcommand \hs{\hat{\sigma}}
\newcommand \htheta{\hat{\theta}}
\newcommand \s{\sigma}
\newcommand \hy{\hat{y}}
\newcommand \hY{\hat{Y}}
\newcommand \e{\varepsilon}
\newcommand \he{\hat{\e}}
\newcommand \cN{\mathcal{N}}


\begin{document}

Можно использовать любой софт с открытым кодом: python, r, julia, gretl, всё, что душе угодно из открытого софта. 
Работу можно выполнять командой из двух человек или в одиночку. Из трёх нельзя. Из четырёх тоже нельзя. 

\begin{enumerate}
    \item «Взять языка». Возьми месячный временной ряд. Можно не один, а несколько, чтобы использовать дополнительные ряды как предикторы основного.
    Не бери цены финансовых инструментов, так как их приращение плохо прогнозируется в силу эффективности рынка.
    Разумно взять реальные показатели. Если очень хочется работать с финансовыми данными, можно взять волатильность, 
    она прогнозируется хорошо. Максимально чётко укажи, откуда взяты ряды. Если ряды парсились, то приведи код. 

    \item «Намалевич». Построй графики рядов, графики автокорреляционных функций, графики с нарезкой ряда на годы для иллюстрации сезонности. 
    
    \item «Твиттер». Кратко прокомментируй полученные графики. 
    Явлются ли ряды сезонными? есть ли тренд? стационарны ли ряды? есть ли точки излома? растёт ли амплитуда колебаний ряда?
    Возьми логарифм ряда, если душа тянется к логарифму.

    % \item «Разделяй и властвуй». Разделите ряды на обучающую и тестовую выборку в пропорции 80\% и 20\%. 

      
    \item «Двенадцать месяцев». Необходимо исследовать качество прогнозов минимум 6 моделей:

    \begin{itemize}
        \item Наивная, $\hat y_{t+1} = y_t$.
        \item Сезонная наивная, $\hat y_{t+1} = y_{t + 1 - 12}$.
        \item SARIMA(1, 1, 1)(1, 0, 0)[12].
        \item Алгоритм Хиндмана-Хандакара автоматического подбора SARIMA.
        \item ETS(AAA)
        \item ETS с автоматическим выбором по AIC.
    \end{itemize}      
    
    Сравни качество прогнозов по средней абсолютной ошибке MAE на тестовой выборке. В качестве тестовой выборки возьми последний год наблюдений,
    в качестве обучающей — все остальные наблюдения. 
    
    На бонусный балл: сравни качество прогнозов по средней абсолютной ошибке MAE на один шаг вперёд с помощью кросс-валидации. 
    Кросс-валидацию проводи растущим окном, начав с окна в 80\% исходной выборки. 

    Подсказка. В питоне могут помочь функции \verb|NaiveForecaster|, \verb|ARIMA|, \verb|AutoARIMA|, \verb|ExponentialSmoothing| и \verb|AutoETS| из \verb|sktime|.
    В r нужное модели есть в пакете \verb|fable|, а графики поможет построить \verb|feasts|. В julia ARIMA и ETS реализованы в пакете \verb|StateSpaceModels|.

    \item «Кто на свете всех милее?» 
    Выбери наилучшую модель и построй график прогнозов для неё на один год вперёд использовав все 100\% наблюдений как обучающую выборку. 


    Ссылки: 
    \begin{itemize}
        \item \verb|sktime|: \url{https://www.sktime.org}.
        \item Прогнозы в sktime, в конце про кросс-валидацию: \url{https://www.sktime.org/en/latest/examples/01_forecasting.html}.
        \item Картинки для кросс-валидации: \url{https://www.kaggle.com/cworsnup/backtesting-cross-validation-for-timeseries}.
        \item Изложение алгоритма Хиндмана-Хандакара, да и вся книжка Forecasting Principles and Practice: \url{https://otexts.com/fpp3/arima-r.html}.
    \end{itemize}

    

    \newpage
    \item «Хождение в народ». Скачай панельные данные RLMS. Можешь выбрать данные по домохозяйствам, а можешь по индивидам, \url{https://www.hse.ru/rlms/}. 
    Регистрация там бесплатная и без смс. Описание огромного количества переменных есть в \url{https://www.hse.ru/rlms/code}.
    К сожалению, данные RLMS выложены в закрытых форматах. Это маленький позор, я с ним пытался бороться, но пока безуспешно.  
    Если уж никак не удаётся загрузить большую панельку из приватного формата сразу в открытый софт, 
    открой её в родном закрытом софте и экспортируй нужные переменные в свободный \verb|.csv|.
    \item «Кому на Руси жить хорошо?». Сформулируй пусть не особо глубокий, но всё же вопрос. 
    Уровня «Помогает ли потребление огурцов домодчадцами предсказать доход главы семейства?» вполне достаточно. 
    В данных RLMS много пропусков, поэтому погоня за большим количество предикторов приведёт к выбрасыванию всех наблюдений.
    Для данного игрового задания одного предиктора достаточно. 
    \item «Сделай красиво!» Визуализируй данные с целью графического ответа на поставленный вопрос. 
    \item «Три сестры». Оцени три модели: сквозную модель, FE-модель, RE-модель. С помощью подходящих тестов выбери наилучшую. 
    
    Подсказка. В питоне панельки оценивают с помощью \url{https://bashtage.github.io/linearmodels/}, в r — \url{https://cran.r-project.org/web/packages/plm/vignettes/A_plmPackage.html},
    в julia — \url{https://nosferican.github.io/Econometrics.jl/stable/} или \url{https://github.com/FixedEffects/FixedEffectModels.jl}.
    \item «Ответ на главный вопрос». Ответь на поставленный вопрос. 

\end{enumerate}

Наставления в добрый путь храброму падавану:

\begin{itemize}
    \item Работу можно сдать в следующих видах. В виде читабельного отчёта в \verb|.pdf|, приложив исходный \verb|.tex|. 
    В виде исполняемого \verb|.ipynb|. В виде \verb|.html| отчёта, сгенерированного из \verb|.Rmd| или \verb|.jmd|, приложив исходник. 

    Ссылки: с \verb|.ipynb| большинство знакомо, поэтому \url{https://github.com/JunoLab/Weave.jl}, \url{https://bookdown.org/yihui/rmarkdown-cookbook/}.

    \item Приложи исходные данные, ряды и нужную для исследования часть панельки, в виде \verb|.csv|. 
    Не надо прикладывать всю скаченную панельку, но только лишь нужную часть. 

    \item Мелкие детали, отсутствующие в условии, заполни самостоятельно, чётко описав свой выбор.
    
    \item Дедлайн: мягкий — 16 июня в 21:00. После мягкого — штраф минус 30\%, жёсткий — 17 июня в 21:00, \url{https://classroom.github.com/a/MKA14xZ7}. 
    
    \item Да пребудет с тобой Сила!
\end{itemize}


\end{document}
