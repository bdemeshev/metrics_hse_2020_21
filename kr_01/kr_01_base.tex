\documentclass[12pt]{article}

\usepackage{tikz} % картинки в tikz
\usepackage{microtype} % свешивание пунктуации

\usepackage{array} % для столбцов фиксированной ширины

\usepackage{indentfirst} % отступ в первом параграфе

\usepackage{sectsty} % для центрирования названий частей
\allsectionsfont{\centering}

\usepackage{amsmath, amssymb, amsthm} % куча стандартных математических плюшек

\usepackage{amsfonts}

\usepackage{comment}

\usepackage[top=2cm, left=1.2cm, right=1.2cm, bottom=2cm]{geometry} % размер текста на странице

\usepackage{lastpage} % чтобы узнать номер последней страницы

\usepackage{enumitem} % дополнительные плюшки для списков
%  например \begin{enumerate}[resume] позволяет продолжить нумерацию в новом списке
\usepackage{caption}


\usepackage{hyperref} % гиперссылки

\usepackage{multicol} % текст в несколько столбцов


\usepackage{fancyhdr} % весёлые колонтитулы
\pagestyle{fancy}
\lhead{Эконометрика, НИУ-ВШЭ}
\chead{контрольная работа №1}
\rhead{2020-10-22}
\lfoot{Вариант $\xi$}
\cfoot{Ни пуха, ни пера!}
\rfoot{\thepage/3}
\renewcommand{\headrulewidth}{0.4pt}
\renewcommand{\footrulewidth}{0.4pt}



\usepackage{todonotes} % для вставки в документ заметок о том, что осталось сделать
% \todo{Здесь надо коэффициенты исправить}
% \missingfigure{Здесь будет Последний день Помпеи}
% \listoftodos - печатает все поставленные \todo'шки


% более красивые таблицы
\usepackage{booktabs}
% заповеди из докупентации:
% 1. Не используйте вертикальные линни
% 2. Не используйте двойные линии
% 3. Единицы измерения - в шапку таблицы
% 4. Не сокращайте .1 вместо 0.1
% 5. Повторяющееся значение повторяйте, а не говорите "то же"



\usepackage{fontspec}
\usepackage{polyglossia}

\setmainlanguage{russian}
\setotherlanguages{english}

% download "Linux Libertine" fonts:
% http://www.linuxlibertine.org/index.php?id=91&L=1
\setmainfont{Linux Libertine O} % or Helvetica, Arial, Cambria
% why do we need \newfontfamily:
% http://tex.stackexchange.com/questions/91507/
\newfontfamily{\cyrillicfonttt}{Linux Libertine O}

\AddEnumerateCounter{\asbuk}{\russian@alph}{щ} % для списков с русскими буквами
\setlist[enumerate, 2]{label=\asbuk*),ref=\asbuk*}

%% эконометрические сокращения
\let\P\relax
\DeclareMathOperator{\Cov}{\mathbb{C}ov}
\DeclareMathOperator{\Corr}{\mathbb{C}orr}
\DeclareMathOperator{\Var}{\mathbb{V}ar}
\DeclareMathOperator{\E}{\mathbb{E}}
\DeclareMathOperator{\P}{\mathbb{P}}
\DeclareMathOperator{\tr}{trace}
\def \hb{\hat{\beta}}
\def \hs{\hat{\sigma}}
\def \htheta{\hat{\theta}}
\def \s{\sigma}
\def \hy{\hat{y}}
\def \hY{\hat{Y}}
\def \v1{\vec{1}}
\def \e{\varepsilon}
\def \he{\hat{\e}}
\def \z{z}
\def \hVar{\widehat{\Var}}
\def \hCorr{\widehat{\Corr}}
\def \hCov{\widehat{\Cov}}
\def \cN{\mathcal{N}}





\def \putyourname{\fbox{
    \begin{minipage}{42em}
      Фамилия, имя, номер группы:\vspace*{3ex}\par
      \noindent\dotfill\vspace{2mm}
    \end{minipage}
  }
}

\def \checktable{
\begin{minipage}{42em}
\begin{tabular}{|m{2cm}|m{2cm}|m{2cm}|m{2cm}|m{2cm}|}
\hline
Тест & 1 &  2 & 3 & Итого \\ \hline
&  &  &  & \\
 &  &   & & \\
 \hline
\end{tabular} $\longleftarrow$ для проверяющего!
\end{minipage}
}

\def \testtable{
\begin{minipage}{42em}
\vspace{4pt}

Ответы на тест:

\vspace{2pt}
\begin{tabular}{|m{1cm}|m{1cm}|m{1cm}|m{1cm}|m{1cm}|m{1cm}|m{1cm}|m{1cm}|m{1cm}|m{1cm}|}
\hline
1 & 2 &  3 & 4 & 5 & 6 & 7 & 8 & 9 & 10 \\ 
\hline
 &  &   &  &  &  &  &  &  &  \\ 
 &  &   &  &  &  &  &  &  &  \\ 
\hline
\end{tabular}
\end{minipage}

}





% [1][3] 1 = one argument, 3 = value if missing
% эта магия создаёт окружение answerlist
% именно в окружении answerlist записаны варианты ответов в подключаемых exerciseXX
% просто \begin{answerlist} сделает ответы в три столбца
% если ответы длинные, то надо в них руками сделать
% \begin{answerlist}[1] чтобы они шли в один столбец
\newenvironment{answerlist}[1][3]{
\begin{multicols}{#1}
\begin{enumerate}[label=\fbox{\emph{\Alph*}},ref=\emph{\alph*}]
}
{
\end{enumerate}
\end{multicols}
}

\newenvironment{answerlist1}{
\begin{enumerate}[label=\fbox{\emph{\Alph*}},ref=\emph{\alph*}]
}
{
\end{enumerate}
}



\excludecomment{solution} % without solutions

\theoremstyle{definition}
\newtheorem{question}{Вопрос}




\begin{document}

% \checktable

% \putyourname

% \testtable

\subsection*{Тест}



\subsection*{Задачи}

\begin{enumerate}


\item Рассеянный исследователь Вовочка 555 дней замерял своё потребление шоколада 
и число решённых задач по эконометрике. 
Вовочка оценил по своим данным парную регрессию числа решенных задач на потребление шоколада 
(регрессию с константой), но потерял все результаты вычислений и не справится без вашей помощи!

\begin{enumerate}
    \item Вовочка запомнил, что 95\%-ый доверительный интервал для коэффициента 
    при шоколаде был от 1.72 до 8.28. Помогите ему восстановить оценку $\hat\beta_{choc}$  и оценку её стандартного отклонения. 
  
    Ошибки в модели для этого и следующего пунктов считайте нормальными.
\item Помогите Вовочке проверить значимость  $\hat\beta_{choc}$  на 10\% уровне значимости.
\item Можно ли было бы считать полученные МНК-оценки коэффициентов несмещёнными и эффективными 
в случае равномерных от -5 до 5 ошибок? Почему?
\item Можно ли было бы считать полученные МНК-оценки коэффициентов несмещёнными и эффективными 
в случае равномерных от 0  до 5 ошибок? Почему?
\end{enumerate}


\item Рассмотрим уравнение линейной регрессии $Y_i = \beta X_i + u_i$. 
Все предпосылки теоремы Гаусса-Маркова выполнены.
\begin{enumerate}
\item Найдите МНК  оценку коэффициента $\hb$.
\item Проверьте,  является  ли эта оценка несмещенной.
\item Выведите формулу для несмещённой оценки дисперсии этой оценки. 
\item По выборке оказалось, что $\hb = 2.25$ и $se(\hb)=0.2$. 
Проинтерпретируйте значение оценки коэффициента.
\item Выведите формулу оценки дисперсии для ошибки прогноза $\hat Y_{N+1}$.
\end{enumerate}

\item Для модели $X_i = \beta_0 + \beta_1 Y_i + u_i$ известна МНК-оценка коэффициента $\hb_1 = -1$. 
Также для данной регрессии известны $N=102$, $\sum (Y_i - \bar Y)^2=10$ и  $TSS=200$. 
\begin{enumerate}
    \item  Найдите коэффициент детерминации $R^2$ для этой модели.
\item Найдите оценку дисперсии оценки коэффициента $\hb_1$.
\item Для регрессии $\hat Y_i = \hat\alpha_0 + \hat\alpha_1 X_i$ найдите оценку $\hat\alpha_1$.
\item Найдите выборочный коэффициент корреляции $\hCorr(X,Y)$.
\end{enumerate}



\end{enumerate}

\newpage

\section{Решения}

\begin{enumerate}
    \item Вместо $t_{553}$-распределения можно использовать нормальное $\cN(0;1)$.
\begin{enumerate}
    \item {[2]} Критическое значение равно $t_{crit} = 1.96$. 

    Отсюда находим $\hb = 5$ и $se(\hb) = 2$
    \item {[1]} $t_{obs} = 5/2 = 2.5$. Таблицы не нужны, достаточно помнить, 
    что при уровне значимости $\alpha=0.05$ критическое значение равно $1.96$. 
    При более высоком уровне значимости критическое значение падает. Значит $H_0$ отвергается. 

    \item {[1]} Ожидание ошибки равно нулю, дисперсия постоянна, значит условия теоремы Гаусса-Маркова выполнены. 
    Обе оценки являются несмещёнными и эффективными среди линейных несмещённых оценок. 

    \item {[2]} Ожидание ошибки равно $2.5$, дисперсия постоянна, значит условия теоремы Гаусса-Маркова нарушены.
    Однако при переносе $2.5$ из ошибки в константу нарушение исчезает. 
    Оценка наклона: несмещена и эффективна среди линейных несмещённых оценок.

    Оценка константы: смещена на 2.5, поэтому оценка не лежит в классе линейных несмещенных оценок, 
    и говорить об её эффективности в этом классе бессмысленно. 
    При этом дисперсия оценки константы равна дисперсии эффективной оценки.
    
\end{enumerate}

\item Кратко:
\begin{enumerate}
    \item {[1]} $\hb = \frac{\sum X_i Y_i}{\sum X_i^2}$
    \item {[1]} $\E(\hb) = \beta$
    \item {[2]} $\Var(\hb) = \frac{\hat\sigma^2_u}{\sum X_i^2}$
    \item {[1]} $t_{obs} = 11.25$, коэффициент значимо отличен от нуля. 
    Зависимая переменная в среднем в $2.25$ раз больше регрессора.
    \item {[2]} $\hVar(\hat Y_{N+1} - Y_{N+1}) = \hat\sigma^2_u \left( 1 + \frac{X_{N+1}^2}{\sum X_i^2}  \right)$
\end{enumerate}

\item Обратите внимание, что $Y_i$ является регрессором. 
\begin{enumerate}
    \item {[2]} $\hb_1 = \frac{\sum (X_i - \bar X)(Y_i - \bar Y)}{\sum (Y_i - \bar Y)^2}$.
    
    Следовательно, $\sum (X_i - \bar X)(Y_i - \bar Y) = -10$. 

    Решаем одним махом г) и а)!

    \[
    \hCorr(X, Y) = \frac{\sum (X_i - \bar X)(Y_i - \bar Y)}{\sqrt{10\cdot 200}} = -\sqrt{5}/10    
    \]

    Отсюда:
    \[
    R^2 = 5/100 = 1/20   
    \]

    \item {[2]} $RSS = 0.95 \cdot 200 = 190$. Отсюда $\hat\sigma^2_u = 190/100 = 1.9$ и $se^2(\hb_1) = 1.9/10=0.19$.
    \item {[2]} В узких кругах широко известно, что корреляции по модулю равна среднему геометрическому оценок 
    в прямой и обратной моделях. 

    \[
    R^2 = \hat\alpha_1 \cdot \hb_1    
    \]

    Следовательно, $\hat\alpha_1 = -1/20$.

    \item {[1]} уже решили!
\end{enumerate}


\end{enumerate}


\end{document}